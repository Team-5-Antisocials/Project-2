\section*{Executive summary}
\addcontentsline{toc}{section}{\protect\numberline{}Executive summary}

The malicious Doodle Jump 2 application seems to be relatively harmless.
From the VirusTotal scan it seemed like this application would be riddled with AdWare, but after having performed a behavior analysis, it became clear that this wasn’t really the case.
What did happen during this analysis is that a pop-up was displayed with a message in Arabic.
Other than this all the collectibles and levels were unlocked.
The network and process analysis did not reveal anything that seemed malicious.
In fact, it showed how similar the application worked with the legitimate application.
The code analyses of both applications did seem to show differences, but not ones which were significant.
All of this would lead an outside observer to believe that the malicious application is not really all that malicious, but rather the result of a hobbyist who likes to tinker with popular apps in his or her free time.
Anybody who would like a fully completed version of an application could then go to his website and download the APK file.
It is safe to say after all these findings that there are not only malicious hackers, but also benevolent ones.

The seemingly innocent DHL application turned out to be very malicious.
VirusTotal indicated that 33/63 security vendors marked the application as malicious.
Most of them indicated that the application was either a Trojan or a Banker.
The application itself had a background of the original application, then proceeded to ask for a lot of permissions.
After which, it redirected to the original DHL application.
The network analysis indicated that the application was sending messages to a telegram user, \texttt{L0gicMan}, using the telegram API.
A more thorough analysis of the code revealed that the application could hack not only almost every bank, but also some crypto wallets and email clients.

The Chrome application was a very recent detection with first having only 16/64 security vendor flags, later on changing to 24/64 flags. 
It was marked by most of the vendors as a Banking Trojan or regular Trojan. 
As soon as the application is installed it will ask the user to enable the app in the application settings giving it free reign over the phone. 
The application will then keep running and not allow the user to uninstall the app. 
It will periodically seek out a connection with a Cloudflare IP sending a Json file. 
The app will also save the users settings and upload them to a local database. 