\section{Scope}

The scope described what data had to be collected, what data was not collected and the reasoning behind these decisions.

For the research only Android applications were tested, the reason behind this was that the necessary tools that were provided only work on Android applications. This means that no IOS applications were tested. Another reason that iOS applications were not tested was that they could not be installed on an emulator on Windows operating systems, while android applications could be installed on Unix systems (also includes MAC OS) and Windows systems.

It was decided that doing a behavior analysis would help to understand how the application works. During this analysis only visual data of the application was collected. The research does not require the collection of any form of background activities performed by the application during the behavior analysis. These activities did not hold value for the behavior analysis.

It was decided that a network analysis was necessary to raise the probability of retrieving the information that the research required. During this analysis all sources that the applications connected to were collected. These sources could consist of IP-addresses and domain names. The research allowed for the gathering of information of any potential servers with the use of tools such as shodan.io. Connecting to these services however fell outside the scope of the research. This had the potential of raising unwanted attention of the malicious parties.

For the proper analysis of an Android application it was decided that doing a code analysis was essential, the code had potential to had revealed a lot of important information about what the application was doing that otherwise had been very hard to find. It had been important for the code analysis to get ahold of the source code of the software. This was achieved by using a decompiling program if the source of the application was not available.

For further understanding of the application, it had been decided that a process analysis was required. During this analysis, the memory and the processes that had been activated in the background were monitored. The different dependencies that the application used, had also been monitored.

During the research there was a possibility that a connection between a malicious application and a hidden database or server were found. This created a chance to potentially gain access to the server or database. It was decided that the actions within the research were within the confounds of the law. For this reason, it was crucial for the research that the international laws were being upheld. One of these laws had prohibited the research from interfering or tampering with the server or database without the express consent of the other party. Because of this reason, it was decided that there would be made notes about the connection and not interact with it in any way.