\section{Conclusion}

Following a set scope and methodology worked out to be a great idea. It made discovering findings from the analyzed APK files relatively easy.
The method applied led to a lot of interesting findings. 
In a lot of cases malicious activity was found. 
This did not mean that there could not be improvements made, especially the use of \texttt{apk-mitm} should have been researched earlier, and should have been specified in the Methodology.
The code analyses revealed that the application creators often did not have good OPSEC skills and their obfuscation would be critically flawed in the most important places.
Revealing important information such as IP's, servers, websites and even names to the analysts using only simple tools and logic.
It was quite interesting to see the variety of applications that contained malicious activity, it really could be any application hacking a bank account.
This proved that the use of an antivirus program is indispensable.
Using commonsense when installing an application is very important.
A few things an android user could think about are:

\begin{itemize}
    \item Only installing applications from the Google Play Store.
    \item Don't do things you're not familiar with, especially when installing applications.
    \item Use the default security measures on your device.
\end{itemize}


