\section{'DHL Express Mobile' Application Analysis by Rafael}

The application that was analyzed in this chapter had the name 'DHL'.
The application was found on \href{https://koodous.com/apks/38ff459a46e9ea6d63a83c1eddb640626fef562cd1bcb0ab3823c4770d07d0fb}{Koodous} with the version \texttt{1.0} and package name \texttt{com.ru.dhl}, the size was \texttt{2.7 MB}.
The same package name did not exist on the Google Play Store. However, an application with the same icon was found \href{https://play.google.com/store/apps/details?id=com.dhl.exp.dhlmobile}{here}. It has the version \texttt{2.7.0}, package name \texttt{com.dhl.exp.dhlmobile} and a size of \texttt{25 MB}.

The application found on Koodous and the application found on the Google Play Store will hereinafter be referred to as 'Malicious application' and 'Original application' respectively.

\subsection{VirusTotal summary}

VirusTotal indicated that the malicious application was marked by 33/63 security vendors as malicious, 
and that the original application was marked by 0/60 security vendors as malicious.

The malicious application was primarily marked as a Banker\footnote{A banker generally is an application that attempts to steal banking information in order to steal the user's money.}. 
If it was not marked as a banker it was either marked as a Trojan\footnote{A Trojan (or 'Trojan Horse malware') is a blanket term for malicious software that disguises itself as a harmless application} or a non repeating name.
This was the full list


\begin{tabular}{ |l|l| }
    \hline
    \textbf{Vendor} & \textbf{Detection} \\
    
    \hline
        Ad-Aware    &   Trojan.GenericKD.37488882 \\
    \hline
        Alibaba     &   TrojanSpy:Android/Banker.66c12705 \\
    \hline
        Antiy-AVL   &   Trojan/Generic.ASMalwAD.5B \\
    \hline
        Arcabit     &   Trojan.Generic.D23C08F2 \\
    \hline
        Avast-Mobile    &   APK:RepSandbox [Trj] \\
    \hline
        Avira (no cloud)    &   ANDROID/Spy.Banker.YD.Gen \\
    \hline
        BitDefender     &   Trojan.GenericKD.37488882 \\
    \hline
        BitDefenderFalx     &   Android.Trojan.Banker.WS \\
    \hline
        CAT-QuickHeal   &   Android.AbereBot.Af9d \\
    \hline
        Cynet   &   Malicious (score: 99) \\
    \hline
        DrWeb   &   Android.BankBot.852.origin \\
    \hline
        Emsisoft    &   Trojan.GenericKD.37488882 (B) \\
    \hline
        eScan   &   Trojan.GenericKD.37488882 \\
    \hline
        ESET-NOD32  &   A Variant Of Android/Spy.Banker.AZU \\
    \hline
        F-Secure    &   Malware.ANDROID/Spy.Banker.YD.Gen \\
    \hline
        FireEye     &   Trojan.GenericKD.37488882 \\
    \hline
        Fortinet    &   Android/AbereBot.A!tr \\
    \hline
        GData   &   Trojan.GenericKD.37488882 \\
    \hline
        Gridinsoft  &   Trojan.U.Banker.oa \\
    \hline
        Ikarus  &   Trojan.AndroidOS.Banker \\
    \hline
        K7GW    &   Spyware ( 005817811 ) \\
    \hline
        Kaspersky   &   HEUR:Trojan-Banker.AndroidOS.AbereBot.a \\
    \hline
        Kingsoft    &   Android.Troj.tn-banker.azu.(kcloud) \\
    \hline
        Lionic  &   Trojan.AndroidOS.AbereBot.C!c \\
    \hline
        MAX     &   Malware (ai Score=100) \\
    \hline
        McAfee  &   Artemis!4778ACA48D17 \\
    \hline
        McAfee-GW-Edition   &   Artemis!Trojan \\
    \hline
        Microsoft   &   TrojanSpy:AndroidOS/Banker.GV!MTB \\
    \hline
        Symantec    &   Trojan.Gen.MBT \\
    \hline
        Symantec Mobile Insight     &   AppRisk:Generisk \\
    \hline
        Tencent     &   A.privacy.AnubisTrojanBanking \\
    \hline
        Trustlook   &   Android.Malware.Trojan \\
    \hline
        ZoneAlarm by Check Point    &   HEUR:Trojan-Banker.AndroidOS.AbereBot.a \\
    \hline
\end{tabular}
\subsubsection{Permission requests}
\subsubsubsection{Malicious application}
The malicious application requested the following permissions:

\texttt{android.permission.ACCESS\_NETWORK\_STATE}
\newline \texttt{android.permission.ACCESS\_WIFI\_STATE}
\newline \texttt{android.permission.CALL\_PHONE}
\newline \texttt{android.permission.CHANGE\_WIFI\_STATE}
\newline \texttt{android.permission.FOREGROUND\_SERVICE}
\newline \texttt{android.permission.INTERNET}
\newline \texttt{android.permission.MODIFY\_AUDIO\_SETTINGS}
\newline \texttt{android.permission.READ\_CALL\_LOG}
\newline \texttt{android.permission.READ\_CONTACTS}
\newline \texttt{android.permission.READ\_PHONE\_STATE}
\newline \texttt{android.permission.READ\_PRIVILEGED\_PHONE\_STATE}
\newline \texttt{android.permission.READ\_SMS}
\newline \texttt{android.permission.RECEIVE\_BOOT\_COMPLETED}
\newline \texttt{android.permission.RECEIVE\_SMS}
\newline \texttt{android.permission.REQUEST\_DELETE\_PACKAGES}
\newline \texttt{android.permission.REQUEST\_IGNORE\_BATTERY\_OPTIMIZATIONS}
\newline \texttt{android.permission.SEND\_SMS}
\newline \texttt{android.permission.SHUTDOWN}
\newline \texttt{android.permission.UPDATE\_DEVICE\_STATS}
\newline \texttt{android.permission.WAKE\_LOCK}
\newline \texttt{android.permission.WRITE\_CALL\_LOG}
\newline \texttt{android.permission.WRITE\_CONTACTS}

\newpage
\subsubsubsection{Original application}
The original application requested the following permissions:

\texttt{android.permission.CALL\_PHONE}
\newline \texttt{android.permission.FLASHLIGHT}
\newline \texttt{android.permission.READ\_APP\_BADGE}
\newline \texttt{android.permission.READ\_EXTERNAL\_STORAGE}
\newline \texttt{android.permission.READ\_PHONE\_STATE}
\newline \texttt{android.permission.USE\_FINGERPRINT}
\newline \texttt{android.permission.WRITE\_EXTERNAL\_STORAGE}
\newline \texttt{com.anddoes.launcher.permission.UPDATE\_COUNT}
\newline \texttt{com.dhl.exp.dhlmobile.permission.C2D\_MESSAGE}
\newline \texttt{com.google.android.c2dm.permission.RECEIVE}
\newline \texttt{com.htc.launcher.permission.READ\_SETTINGS}
\newline \texttt{com.htc.launcher.permission.UPDATE\_SHORTCUT}
\newline \texttt{com.huawei.android.launcher.permission.CHANGE\_BADGE}
\newline \texttt{com.huawei.android.launcher.permission.READ\_SETTINGS}
\newline \texttt{com.huawei.android.launcher.permission.WRITE\_SETTINGS}
\newline \texttt{com.majeur.launcher.permission.UPDATE\_BADGE}
\newline \texttt{com.oppo.launcher.permission.READ\_SETTINGS}
\newline \texttt{com.oppo.launcher.permission.WRITE\_SETTINGS}
\newline \texttt{com.sec.android.provider.badge.permission.READ}
\newline \texttt{com.sec.android.provider.badge.permission.WRITE}
\newline \texttt{com.sonyericsson.home.permission.BROADCAST\_BADGE}
\newline \texttt{com.sonymobile.home.permission.PROVIDER\_INSERT\_BADGE}
\newline \texttt{me.everything.badger.permission.BADGE\_COUNT\_READ}
\newline \texttt{me.everything.badger.permission.BADGE\_COUNT\_WRITE}
\subsection{Behavior analysis}

When the malicious application was opened it showed a popup telling the user that the malicious application required accessability features to be turned on.
It also, gave instructions on how to turn the accessability features on.
After clicking OK it opens settings on the page where the accessability features can be turned on. 
When the accessability features were turned on the malicious application took over, closed all the prompts given by the settings page and brought the user back to the malicious application.

Inside the malicious application a new popup was waiting, this one told the user to give the malicious application notification access. After pressing okay it again opened the settings, but this time on the notification settings page. After granting the application these permissions it again closes all prompts. 
This time however it closed the malicious application as well.
When the malicious application closed a popup appeared telling the user that the malicious application kept stopping.
However, the application was still present in the application tray.

When the application was clicked on inside the application tray it redirected the user to the Google Play Page of the legitimate \href{https://play.google.com/store/apps/details?id=com.dhl.exp.dhlmobile&hl=en&gl=US}{DHL application}.
The url was the same as the original application found earlier, with language parameters added.
If the application was closed inside the application tray and then again reopened via the shortcut created after installing it also redirected to the same page.





\newpage
\subsection{Network analysis}

<what does the network traffic look like>

\subsubsection{HTTP proxy analysis}

<what did the proxy reveal about the network traffic>

\subsubsection{Wireshark analysis}

<what did WireShark reveal about the network traffic>

\subsubsection{Reconnaissance [optional]}

<what was found using shodan.io>

\newpage
\subsection{Code analysis}

<interesting stuff found in code>

\newpage
\subsection{Process analysis}

<interesting stuff found in the process and memory, etc.>

\newpage
\subsection{Countermeasures and detection}

\subsubsection{Preventive measures}


<Measures to prevent installing this application>

\subsubsection{Detective measures}

<Measures to detect if this application is installed>

\subsubsection{Reactive measures}

<What to do if it is installed>

\subsubsection{YARA rule set}

<A YARA rule set created based on your findings>
