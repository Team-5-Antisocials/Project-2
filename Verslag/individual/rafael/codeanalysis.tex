\subsection{Code analysis}

The malicious code was successfully decompiled using JadX, there were quite a lot of elements it was not able to decompile.
However, it seemed like the most important parts were intact. 
The code was obfuscated. The connection between classes was impossible to determine.
However, there were still a lot of useful strings and android interactions.

One of the interesting strings existed in two files, \texttt{sources/d/e/a/b.java} and \texttt{sources/d/c/a/b/a.java}.
The strings contained the domain that requests were being sent to, \texttt{https://api.telegram.org/bot}.
This indicated that the application really was making requests to Telegram.

There was a file that was a lot more interesting, however, \texttt{sources/d/e/a/e.java}.
This file contained a long list of package names, which on a first quick glance looked like package names for popular crypto banking applications.
After some investigating it not only contained crypto banking applications but also regular banking applications and email clients.
It even included Dutch banks such as IGN, ABN AMRO and a few others. 

A list of all the applications in the list can be found in Table \ref{rafael-hackeableapps} included at the end of this section.
Some further investigation indicated that the application was attempting to steal data from these applications and send it somewhere.
It can be assumed that it again sends the data to telegram. However, it was near impossible to find in the code whether this was actually the case.

\newpage
    \begin{longtblr}[
        caption = {All applications that can be hacked},
        label = {rafael-hackeableapps}
    ]{
        colspec = {|l|l|},
        rowhead = 1,
        hlines
    }
    \textbf{Google Play Store application}                & \textbf{Package name}                      \\
                                                          & io.hotbit.shouy                            \\
                                                          & localbitcoin                               \\
                                                          & pl.ideabank.mobilebanking                  \\
    ABN AMRO                                              & com.abnamro.nl.mobile.payments             \\
    Albaraka Mobile Banking                               & com.albarakaapp                            \\
    Alior Mobile                                          & pl.aliorbank.aib                           \\
    Allegro - convenient and secure online   shopping     & pl.allegro                                 \\
    ANZ Australia                                         & com.anz.android.gomoney                    \\
    AOL - News, Mail \& Video                             & com.aol.mobile.aolapp                      \\
    ASN Mobiel Bankieren                                  & nl.asnbank.asnbankieren                    \\
    Axis Mobile - Fund Transfer,UPI,Recharge   \& Payment & com.axis.mobile                            \\
    Banca MPS                                             & it.copergmps.rt.pf.android.sp.bmps         \\
    Banca Móvil Laboral Kutxa                             & com.tecnocom.cajalaboral                   \\
    Bank Millennium                                       & wit.android.bcpBankingApp.millenniumPL     \\
    Bank Millennium for Companies                         & pl.millennium.corpApp                      \\
    Bank of America Mobile Banking                        & com.infonow.bofa                           \\
    Bank of Melbourne Mobile Banking                      & org.bom.bank                               \\
    Bank of Scotland Mobile Banking                       & com.grppl.android.shell.BOS                \\
    Bankia                                                & es.cm.android                              \\
    Bankinter Móvil                                       & com.bankinter.launcher                     \\
    BankSA Mobile Banking                                 & org.banksa.bank                            \\
    Bankwest                                              & au.com.bankwest.mobile                     \\
    Barclays                                              & com.barclays.android.barclaysmobilebanking \\
    BBVA Net Cash | ES \& PT                              & com.bbva.netcash                           \\
    BCU Mobile Banking                                    & com.bcu.bcu                                \\
    Bendigo Bank                                          & com.bendigobank.mobile                     \\
    BHIM UPI, Money Transfer, Recharges \&   Pay Later    & com.mobikwik\_new                          \\
    Bill Payment \& Recharge,Wallet                       & com.oxigen.oxigenwallet                    \\
    Binance: BTC NFTs Memes \& Meta                       & com.binance.dev                            \\
    Bitfinex: Trade Digital Assets                        & com.bitfinex.mobileapp                     \\
    Bithumb                                               & com.btckorea.bithumb                       \\
    Blockchain.com Wallet - Buy Bitcoin, ETH,   \& Crypto & piuk.blockchain.android                    \\
    BNL                                                   & it.bnl.apps.banking                        \\
    BNL PAY                                               & it.bnl.apps.enterprise.bnlpay              \\
    bob World                                             & com.bankofbaroda.mconnect                  \\
    BOQ Mobile                                            & com.bankofqueensland.boq                   \\
    BPS Mobilnie                                          & pl.bps.bankowoscmobilna                    \\
    BtcTurk | PRO - Buy-Sell Bitcoin                      & com.btcturk.pro                            \\
    CA24 Mobile                                           & com.finanteq.finance.ca                    \\
    CaixaBankNow                                          & es.lacaixa.mobile.android.newwapicon       \\
    Capital One Mobile                                    & com.konylabs.capitalone                    \\
    Ceneo - zakupy i promocje                             & pl.ceneo                                   \\
    CEPTETEB                                              & com.teb                                    \\
    Chase Mobile                                          & com.chase.sig.android                      \\
    CIBC Mobile Banking®                                  & com.cibc.android.mobi                      \\
    Citi Mobile®                                          & com.citi.citimobile                        \\
    Citibank Australia                                    & com.citibank.mobile.au                     \\
    Coinbase: Buy BTC, Ethereum, SHIB,   Bitcoin Cash     & com.coinbase.android                       \\
    comdirect mobile App                                  & de.comdirect.android                       \\
    Commerzbank Banking - The app at your   side          & de.commerzbanking.mobil                    \\
    Consorsbank                                           & de.consorsbank                             \\
    Deutsche Bank Mobile                                  & com.db.pwcc.dbmobile                       \\
    DKB-Banking                                           & de.dkb.portalapp                           \\
    Empik                                                 & com.empik.empikapp                         \\
    Empik Foto                                            & com.empik.empikfoto                        \\
    Enpara.com Cep Şubesi                                 & finansbank.enpara                          \\
    ESL Mobile Banking                                    & com.ifs.banking.fiid3364                   \\
    EVO Banco móvil                                       & es.evobanco.bancamovil                     \\
    Fifth Third Mobile Banking                            & com.clairmail.fth                          \\
    First Tech Federal CU                                 & com.firsttech.firsttech                    \\
    Garanti BBVA Mobile                                   & com.garanti.cepsubesi                      \\
    Getin Mobile                                          & com.getingroup.mobilebanking               \\
    Gmail                                                 & com.google.android.gm                      \\
    GOmobile Biznes                                       & com.mobile.banking.bnp                     \\
    GOPAX (Crypto exchange)                               & kr.co.gopax                                \\
    Great Southern Bank Australia                         & au.com.cua.mb                              \\
    Halifax Mobile Banking                                & com.grppl.android.shell.halifax            \\
    Halkbank Mobil                                        & com.tmobtech.halkbank                      \\
    HDFC Bank MobileBanking App                           & com.snapwork.hdfc                          \\
    HSBC Turkey                                           & tr.com.hsbc.hsbcturkey                     \\
    HSBC UK Mobile Banking                                & uk.co.hsbc.hsbcukmobilebanking             \\
    HSBC US                                               & us.hsbc.hsbcus                             \\
    HVB Mobile Banking                                    & eu.unicreditgroup.hvbapptan                \\
    iBiznes24 mobile                                      & pl.bzwbk.ibiznes24                         \\
    IDBI Bank mPassbook                                   & com.idbi.mpassbook                         \\
    IKO                                                   & pl.pkobp.iko                               \\
    IMB.Banking                                           & com.imb.banking2                           \\
    imo video calls and chat                              & com.imo.android.imoim                      \\
    iMobile Pay by ICICI Bank                             & com.csam.icici.bank.imobile                \\
    IndOASIS - Indian Bank Mobile Banking                 & com.IndianBank.IndOASIS                    \\
    ING Australia Banking                                 & au.com.ingdirect.android                   \\
    ING Bankieren                                         & com.ing.mobile                             \\
    ING Banking to go                                     & de.ingdiba.bankingapp                      \\
    ING Business                                          & com.comarch.security.mobilebanking         \\
    ING Italia                                            & it.ingdirect.app                           \\
    ING Mobil                                             & com.ingbanktr.ingmobil                     \\
    Instagram                                             & com.instagram.android                      \\
    Intesa Sanpaolo Mobile                                & com.latuabancaperandroid                   \\
    iPKO biznes                                           & pl.pkobp.ipkobiznes                        \\
    İşCep - Mobile Banking                                & com.pozitron.iscep                         \\
    Katılım Mobil                                         & com.ziraatkatilim.mobilebanking            \\
    korbit                                                & com.korbit.exchange                        \\
    KuCoin: Bitcoin, Crypto Trade                         & com.kubi.kucoin                            \\
    Kutxabank                                             & com.kutxabank.android                      \\
    Kuveyt Türk Mobile                                    & com.kuveytturk.mobil                       \\
    Lloyds Bank Mobile Banking                            & com.grppl.android.shell.CMBlloydsTSB73     \\
    mail.com Mail \& Cloud                                & com.mail.mobile.android.mail               \\
    Mail.ru - Email App                                   & ru.mail.mailapp                            \\
    MB+                                                   & it.bpc.proconl.mbplus                      \\
    mBank PL                                              & pl.mbank                                   \\
    ME Bank                                               & au.com.mebank.banking                      \\
    Messenger                                             & com.facebook.orca                          \\
    Microsoft Outlook                                     & com.microsoft.office.outlook               \\
    MKB Online                                            & ru.mkb.mobile                              \\
    MobilDeniz                                            & com.denizbank.mobildeniz                   \\
    Mobile Banking UniCredit                              & com.unicredit                              \\
    Mobile.UniCredit                                      & ru.ucb.android                             \\
    My Verizon                                            & com.vzw.hss.myverizon                      \\
    myAT\&T                                               & com.att.myWireless                         \\
    Mycelium Bitcoin Wallet                               & com.mycelium.wallet                        \\
    Mój Orange                                            & pl.orange.mojeorange                       \\
    NAB Mobile Banking                                    & au.com.nab.mobile                          \\
    Nationwide Banking App                                & co.uk.Nationwide.Mobile                    \\
    NatWest Mobile Banking                                & com.rbs.mobile.android.natwest             \\
    Navy Federal Credit Union                             & com.navyfederal.android                    \\
    NETELLER - fast, secure and global money   transfers  & com.moneybookers.skrillpayments.neteller   \\
    norisbank App                                         & com.db.mm.norisbank                        \\
    Nowbanking                                            & it.gruppocariparma.nowbanking              \\
    Odeabank                                              & com.magiclick.odeabank                     \\
    OK: Social Network                                    & ru.ok.android                              \\
    Papara                                                & com.mobillium.papara                       \\
    Paxful Bitcoin \& Crypto Wallet | Buy   BTC ETH USDT  & com.paxful.wallet                          \\
    Pay – die App der Volksbanken   Raiffeisenbanken      & de.fiduciagad.android.vrwallet             \\
    PAYEER                                                & com.payeer                                 \\
    Payoneer – Global Payments Platform for   Businesses  & com.payoneer.android                       \\
    PayPal - Send, Shop, Manage                           & com.paypal.android.p2pmobile               \\
    People's Choice Credit Union                          & com.fusion.ATMLocator                      \\
    Perfect Money                                         & com.touchin.perfectmoney                   \\
    Plus500: CFD Online Trading on Forex and   Stocks     & com.Plus500                                \\
    PNC Mobile                                            & com.pnc.ecommerce.mobile                   \\
    Poloniex Crypto Exchange                              & com.plunien.poloniex                       \\
    Postbank Finanzassistent                              & de.postbank.finanzassistent                \\
    Postepay                                              & posteitaliane.posteapp.apppostepay         \\
    ProtonMail - Encrypted Email                          & ch.protonmail.android                      \\
    QIWI Wallet                                           & ru.mw                                      \\
    QNB Finansbank                                        & com.finansbank.mobile.cepsube              \\
    Raiffeisen Online Russia                              & ru.raiffeisennews                          \\
    RB Online                                             & ru.rosbank.android                         \\
    RBC Mobile                                            & com.rbc.mobile.android                     \\
    Regions Bank                                          & com.regions.mobbanking                     \\
    Rossmann PL                                           & pl.com.rossmann.centauros                  \\
    Royal Bank of Scotland Mobile Banking                 & com.rbs.mobile.android.rbs                 \\
    ruralvia                                              & com.rsi                                    \\
    Santander                                             & es.bancosantander.apps                     \\
    Santander Banking                                     & de.santander.presentation                  \\
    Santander mobile                                      & pl.bzwbk.bzwbk24                           \\
    Santander Mobile Banking                              & uk.co.santander.santanderUK                \\
    Scotiabank Mobile Banking                             & com.scotiabank.banking                     \\
    SCRIGNOapp                                            & it.popso.SCRIGNOapp                        \\
    SDFCU Mobile Banking                                  & com.ifs.banking.fiid8025                   \\
    Skrill - Fast, secure online payments                 & com.moneybookers.skrillpayments            \\
    Snapchat                                              & com.snapchat.android                       \\
    SNS Mobiel Betalen                                    & nl.snsbank.mobielbetalen                   \\
    SpardaApp                                             & de.sdvrz.ihb.mobile.app                    \\
    Sparkasse Ihre mobile Filiale                         & com.starfinanz.smob.android.sfinanzstatus  \\
    St.George Mobile Banking                              & org.stgeorge.bank                          \\
    Stripe Dashboard                                      & com.stripe.android.dashboard               \\
    Suncorp Bank                                          & au.com.suncorp.SuncorpBank                 \\
    SunTrust Mobile App                                   & com.suntrust.mobilebanking                 \\
    TARGOBANK Mobile Banking                              & com.targo\_prod.bad                        \\
    TD Bank (US)                                          & com.tdbank                                 \\
    TD Canada                                             & com.td                                     \\
    Telegram                                              & org.telegram.messenger                     \\
    Tesco Bank Mobile Banking                             & com.tescobank.mobile                       \\
    Tinkoff                                               & com.idamob.tinkoff.android                 \\
    Triodos Bankieren NL                                  & com.triodos.bankingnl                      \\
    Trust: Crypto \& Bitcoin Wallet                       & com.wallet.crypto.trustapp                 \\
    TSB Mobile Banking                                    & uk.co.tsb.newmobilebank                    \\
    Tutu.ru - flights, Russian railway and   bus tickets  & ru.tutu.tutu\_emp                          \\
    U-Mobile - Union Bank of India                        & com.infrasoft.uboi                         \\
    U.S. Bank - Secure and easy mobile   banking          & com.usbank.mobilebanking                   \\
    UBI Banca                                             & it.nogood.container                        \\
    Ulster Bank NI Mobile Banking                         & com.rbs.mobile.android.ubn                 \\
    Unocoin- India’s First Bitcoin \&   Crypto Exchange   & com.unocoin.unocoinwallet                  \\
    USAA Mobile                                           & com.usaa.mobile.android.usaa               \\
    VakıfBank Mobil Bankacılık                            & com.vakifbank.mobile                       \\
    VTB-Online                                            & ru.vtb24.mobilebanking.android             \\
    VyStar Mobile Banking                                 & org.vystarcu.mobilebanking                 \\
    Wells Fargo Mobile                                    & com.wf.wellsfargomobile                    \\
    Western Union: Send Money Fast                        & com.westernunion.android.mtapp             \\
    Westpac                                               & org.westpac.bank                           \\
    WhatsApp Messenger                                    & com.whatsapp                               \\
    Woodforest Mobile Banking                             & com.woodforest                             \\
    Yahoo Mail – Organized Email                          & com.yahoo.mobile.client.android.mail       \\
    Yandex Go — taxi and delivery                         & ru.yandex.taxi                             \\
    Yapı Kredi Mobile                                     & com.ykb.android                            \\
    Yono Lite SBI - Mobile Banking                        & com.sbi.SBIFreedomPlus                     \\
    YONO SBI: The Mobile Banking and   Lifestyle App!     & com.sbi.lotusintouch                       \\
    YouApp                                                & com.lynxspa.bancopopolare                  \\
    Ziraat Mobile                                         & com.ziraat.ziraatmobil                     \\
    ŞEKER MOBİL                                           & tr.com.sekerbilisim.mbank                  \\
    Ак Барс Онлайн                                        & ru.akbars.mobile                           \\
    Альфа-Банк (Alfa-Bank)                                & ru.alfabank.mobile.android                 \\
    Альфа-Бизнес                                          & ru.alfabank.oavdo.amc                      \\
    Банк Авангард                                         & ru.avangard                                \\
    Банк Открытие                                         & com.openbank                               \\
    Восточный мобайл                                      & ru.ftc.faktura.expressbank                 \\
    Мобильный банк (старый)                               & com.idamobile.android.hcb                  \\
    Мобильный банк, Россельхозбанк                        & ru.rshb.dbo                                \\
    Модульбанк - банк для вашего бизнеса                  & modulbank.ru.app                           \\
    МТС Банк                                              & ru.mts.money                               \\
    ПСБ                                                   & logo.com.mbanking                          \\
    СберБанк Онлайн — с Салютом                           & ru.sberbankmobile                          \\
    Телекард 2.0                                          & ru.gazprombank.android.mobilebank.app      \\
    비트코인 거래소\&지갑 유빗(Youbit) (This is korean)                               & com.ddengle.bts                            \\
    업비트 - 가장 신뢰받는 디지털 자산(비트코인, 이더리움, 비트코인캐시)   거래소(Also korean)         & com.dunamu.exchange                        \\
    코인원 - Coinone       (Also korean)                                  & coinone.co.kr.official                     \\
    \end{longtblr}