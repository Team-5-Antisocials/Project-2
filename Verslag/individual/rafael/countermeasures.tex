\subsection{Countermeasures and detection}

\subsubsection{Preventive measures}

It was quite easy to prevent yourself from installing this malicious application.
The only thing necessary to so is to only install apps from the Google Play Store.

The original application exists on the Google Play Store and can easily be found there.
As the original application is widely used it will always be suggested over any malicious versions of the original application.

\subsubsection{Detective measures}

To detect if the application has been installed on an android device it is important to go to the settings of the device.
On the applications' page it will clearly show any installed applications.
Since the app does not hide itself from this page one only needs to look through the list to see if any application with the name \texttt{DHL} is installed.
Any variations of this name are not this application, for example \texttt{DHL Express Mobile} would not be this application.

Additionally, it is always advisable to have an antivirus installed,
specifically \href{https://play.google.com/store/apps/details?id=com.antivirus&hl=en&gl=US}{AVG AntiVirus} is a widely used and trusted antivirus,
it has been installed over 100 million times and has a rating of \texttt{4.7\textbackslash5} with 7.3 million reviewers.

\subsubsection{Reactive measures}

If the application has been found on a device it is very important to immediately uninstall the application.
Additionally, it is very possible that if any of the applications described in Table \ref{rafael-hackeableapps} are installed on the device that its information has been stolen.
This could include passwords, usernames, banking information and more.

It is important to contact the correct authorities for each application. 
The contact information for these authorities can often be found inside the application or on the website of the creator of the application.

For additional support it could be useful to contact your local authorities on the non-emergency number.

\newpage
\subsubsection{YARA rule set}
\lstinputlisting[language=YARA]{individual/rafael/ruleset.yar}
% \input{individual/rafael/ruleset.yar}
% somehow include the yara ruleset in the document. If possible with highlighting???