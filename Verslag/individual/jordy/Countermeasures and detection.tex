\subsection{Countermeasures and detection}
A part of malware analysis is coming up with countermeasures and ways to detect the application.
The different countermeasures are described in this chapter.

\subsubsection{Preventive measures}
There were a few preventive measures a user could take to not install this malicious application.
One of these was to only install applications that are on the official Google Play Store.
The only Firefox application that is downloadable on the Google Play Store is the official application from Mozilla.
This would mean that the malicious application is not able to be installed on an user’s phone via the Google Play Store.

The other preventive measure was to install an antivirus software on a user’s phone.
The antivirus software would flag the malicious application as malware and would give a notification or block the installation all together.

\subsubsection{Detective measures}
The best way to detect that the malicious application was installed was by checking the notification bar.
The application always showed up in the notification bar as “FireFox” with “Browser” underneath it.
When this notification was clicked on, it showed nothing.

Another way to detect the malicious application is by looking in the settings app and checking the apps tab.
In this part of settings all installed apps are shown.
The malicious application said it was not installed but showed up in this list.
This meant that the application was indeed installed.

The last described detective measure is installing an antivirus software.
Many security vendors including AVG AntiVirus showed on VirusTotal that the application was malicious.
The antivirus software would indicate that the application is malicious and would give the user a warning.

\subsubsection{Reactive measures}
When the application is detected on a device it is important to immediately uninstall the application.
With the permissions the application asks for it is possible that it send a lot of information to the command and control server and it could do a lot of damage to the Android installation on the phone.
Since the malicious application has the permission to install packages it would be advised to check all installed apps and uninstall any apps the user does not know.
The best solution would be to do a factory reset of the phone.

\subsubsection{YARA rule set}
A YARA rule set is intended to automatically find a specific group of applications.
The designed YARA ruleset that is shown below is designed to specifically find this malicious application using the start of the classes.dex hex code.

\lstinputlisting[language=YARA]{individual/jordy/ruleset.yara}