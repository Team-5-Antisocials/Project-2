\section{Firefox Application Analysis by Jordy}
The malicious application that was analyzed in this chapter had the name "FireFox".
This application was found on \href{https://koodous.com/apks/26a7576cc1182bf90fb16c3320d12a736b3faa10c158755605f36daae4b197b7}{Koodous} with the package name “com.eset.ems2.gp” and with version number 85.1.3.
The size of this application was 70.4 MB.
To confirm that this application was actually malicious a search on the \href{https://play.google.com/store/apps/details?id=org.mozilla.firefox&hl=nl&gl=US}{Google Play Store} was performed.
The current official version of Mozilla Firefox is 94.1.2.
To get the official Firefox version 85.1.3 it was necessary to download the application from another source.
The APK file was checked with \href{https://www.virustotal.com/gui/file/59ce0f9ea256b4576f391d01c685ced2db224a252bf09c3f362e6859a6c7ead5/details}{VirusTotal} and compared to the most recent official version of Mozilla Firefox.
This process has been described in detail in chapter 4.1.

The official APK file for Mozilla Firefox version 85.1.3 had the package name “org.mozilla.firefox” and had the size of 63.39 MB.
All aspects were relatively similar been the versions 85.1.3 and 94.1.2 of Mozilla Firefox with a few small alterations.
These alterations were because one of the versions was a few months older.

From this moment onwards in the chapter the “official application” will be referring to the 85.1.3 version of Mozilla Firefox with the package name “org.mozilla.firefox” and the “malicious application” will be referring to the APK file found on Koodous with the package name “com.eset.ems2.gp”.