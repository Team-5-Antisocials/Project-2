\subsection{Behavior analysis}

As soon as the application was installed it would constantly give a popup telling the user to enable Chrome. 
It also would constantly open a window showing the user how to "enable Chrome". 
If the user tries to force the app to close via these settings it will not work. 
The app will constantly open and force the user to enable the app until it is either uninstalled or enabled.
 
Touching the popup would redirect the user to the Accessibility settings. 
In this menu it would be possible to enable the malicious app. 
What it would enable is however not shown on these settings.

If the user enables this setting the application would disappear from the main menu and silently run in the background. 
According to the phone the app would take a lot of battery life, because it would constantly give popups asking to close background applications to save battery power. 
These popups would however be instantly closed by the application.

The only way the user would be able to find this app would be in the application settings. 
Any attempts from the user to remove the app at this stage will be thwarted by the app.
This is shown on figure \ref{tim-appbehavior}

\begin{figure}[H]
    \centering
    \includegraphics[width=4cm, height=15cm, keepaspectratio]{behaviorimg.png}
    \caption{app booting you from application settings}
    \label{tim-appbehavior}
\end{figure}

The message that the user will be met with when trying to edit the application settings of the malicious chrome app translates to: 
Chrome is working to update your phone please try again later.

According to the translater this was Turkish hinting that this might be made by a Turkish developer.
If the application has reached this stage it would be impossible to remove it from a normal phone.
