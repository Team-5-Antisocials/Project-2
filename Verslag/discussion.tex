\section{Discussion}
In this chapter, the team members looked back on the entire research and performed an evaluation.
This chapter describes what aspects went well, what aspects went wrong and what the learning moments were within this research.

\subsection{What went well?}
During the documentation process for the scope and methodology chapters the team looked back on earlier received feedback and tried to implement it.
After a chapter was finished the team went back to get more feedback and asked for the argumentation behind it.
With this feedback and the reasoning behind it, it became clear what went wrong and how it could be improved on during the entire duration of the research.
From that moment onwards basic documentation rules were written down and used as a baseline for the entire document.

In a fairly short amount of time the team found relevant and sometimes shocking results about the malicious applications.
The analysis methods were performed correctly and based on the finding the team member could give an extensive look into the different aspects of the malicious applications.

\subsection{What went wrong?}
The process of looking for malicious applications did not go well for all team members.
It took quite some time to find applications that matched the requirements for this research.
Many previously selected applications turned out to not be malicious.
This resulted in a lesser amount of time for team members to research the different aspects of their application.

During the network analysis mitmproxy was used to capture network from the malicious applications.
During the analysis team members encountered a few TLS errors.
The team’s preparations on using mitmproxy was lacking.
This resulted in team members taking quite some time looking into these errors.

The use of the different analysis tools did not go smoothly for all team members.
During the analysis team members got stuck on a problem and tried to solve it themselves.
This often took longer than intended.
Other team members were able to help but this help was asked to late.

\newpage
\subsection{Learning moments}
One of the learning moments was to think better about some aspects of the methodology.
A way to improve the search for malicious applications was to look at the selected applications as a team and evaluate them together.
If this would have been included in the methodology it would have saved team members time from evaluating applications that clearly did not match the requirements. 

Another learning moment was during the analysis phase of the research.
Team members were stuck on a problem and tried to solve this on their own.
This resulted in team members losing valuable time that was not necessary since team members could have offered their help.
Team member should ask for help sooner if they are stuck on a problem for an extended amount of time.