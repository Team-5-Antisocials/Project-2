\section{Introduction}

The research group has analyzed one malicious Android application per person. 
For this research the Android application was analyzed in various ways. 
This research was performed in order to gain more practical experience analyzing malicious software. 
The work done in this project has educated the team on how to safely deal with malicious software. 
It has also served as a practice environment to get familiar with the tools needed to analyze these applications. 

This research was done on Android studio working with a virtual machine to make sure the team members were safe while analyzing the application. 
Working with malicious software safety is never a guarantee, so it is in the analyzers best interest to be careful when coming into contact with malware.

This rapport starts with describing the scope of the project in chapter 2. 
In chapter 3 describes the teams’ methodology. 
Chapter 4 contains the application analysis FireFox by Jordy. 
Chapter 5 contains the application analysis Doodle Jump done by Michael. 
Chapter 6 has the application analysis of ‘DHL Express Mobile’ done by Rafael. 
And in Chapter 7 the application analysis of ‘Chrome’ has been done by Tim. 
Chapter 8 has the Discussion and chapter 9 is the conclusion of this rapport.
